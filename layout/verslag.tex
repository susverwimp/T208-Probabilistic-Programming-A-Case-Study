%% Indien je niet vertrouwd ben met Latex:
%%  Maak een .pdf als volgt:
%%  - Vul alles in 
%%  - Doe: pdflatex verslag.tex (dit produceert de .pdf)

\documentclass[12pt]{report}
%\usepackage{a4wide}

\setlength{\parindent}{0cm}

\begin{document}
\pagestyle{myheadings}
\markright{Tussentijds verslag November/Maart -  Student(en): naam ?} % 1) "November/Maart": schrap wat niet past.
                                                                      % 2) Vul je naam in !
(Enkel in november:) {\bf Titel eindwerk:} {\em ?}

\vspace{0.5cm}
(Enkel in maart:) {\bf Definitieve titel eindwerk (na overleg met promotor(s)/begeleider(s)):}
\begin{itemize}
\item {\bf in het Nederlands:} {\em ?}
\item {\bf in het Engels:} {\em ?}
\end{itemize}


\vspace{0.5cm}
{\bf Promotor(s):} ?


\vspace{0.5cm}
{\bf Begeleider(s):} ?

\vspace{1cm}
{\bf Korte situering en Doelstelling: } 
?

\vspace{1cm}
{\bf Belangrijkste bestudeerde literatuur:}
\begin{itemize}
\item ?
\item ?
\item ?
\end{itemize}

\vspace{1cm}
{\bf Geleverd werk (inclusief tijdsrapportering):}
?

\vspace{1cm}
{\bf Belangrijkste resultaten:}
?


\vspace{1cm}
{\bf Belangrijkste moeilijkheden:}
?

\vspace{1cm}
{\bf Gepland werk:} 
?


\vspace{1cm}
{\bf Als ik verder werk zoals ik tot nu toe deed, dan denk ik ??/20 te
    verdienen op het einde.}

{\bf Ik plan mijn eindwerk af te geven in januari/juni/augustus} (schrap wat niet past) 


\end{document}
