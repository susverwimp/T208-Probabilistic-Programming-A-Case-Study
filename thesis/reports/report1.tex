%% Indien je niet vertrouwd ben met Latex:
%%  Maak een .pdf als volgt:
%%  - Vul alles in 
%%  - Doe: pdflatex verslag.tex (dit produceert de .pdf)

\documentclass[12pt]{report}
%\usepackage{a4wide}

\setlength{\parindent}{0cm}

\begin{document}
\pagestyle{myheadings}
\markright{Tussentijds verslag November -  Student: Sus Verwimp} % 1) "November/Maart": schrap wat niet past.
                                                                      % 2) Vul je naam in !
{\bf Titel eindwerk:} {\em Programmeren met kansen: een case study}


\vspace{0.5cm}
{\bf Promotor:} Tom Schrijvers


\vspace{0.5cm}
{\bf Begeleider:} Alexander Vandenbroucke

\vspace{1cm}
{\bf Korte situering en Doelstelling: } 
Mijn thesis situeert zich rond het programmeren met kansen, meer bepaald het vergelijken en evalueren van verschillende Probabilistic Programming languages aan de hand van qualitatieve en quantitatieve criteria.

\vspace{1cm}
{\bf Belangrijkste bestudeerde literatuur:}
\begin{itemize}
\item Design and Implementation of Probabilistic Programming Language Anglican
\item Exploiting Local and Repeated Structure in Dynamic Bayesian Networks
\item Inference and Learning in Probabilistic Logic Programs using Weighted Boolean Formulas
\item Probabilistic Logic Programming Concepts
\item Bayesian Reasoning and Machine Learning
\item Design and Implementation of Probabilistic Programming Language Anglican
\item http://www.robots.ox.ac.uk/~fwood/anglican//2016/12/02/PPAMLSS2016/
\end{itemize}

\vspace{1cm}
{\bf Geleverd werk (inclusief tijdsrapportering):}
\begin{itemize}
\item Eerste 2 weken: Prolog herhaalt en Problog bestudeert. Tijdens deze 2 weken heb ik ook een spel bedacht om te modelleren. 
\item Week 3: Model geimplementeerd om de SUCCESS kansen te bereken van verschillende bord configuraties van het spel.
\item Week 3-4: literatuurstudie over inferentiemethodes en algorimes.
\item week 5: literatuurstudie en oefeningen over Anglican en voorbereiding voor de eerste presentatie.
\item week 6: Nieuw model geschreven dat gebruik maakt van spel strategi\"{e}n waar duidelijk de MARG en MPE kan berekent worden. Implementeren van het nieuwe model in ProbLog en Anglican.
\end{itemize}

\vspace{1cm}
{\bf Belangrijkste resultaten:}
Op het moment zijn de resultaten miniem. Het berekenen van SUCCES kansen voor verschillende bord configuraties werkt. Omdat het model nog niet geschreven is in Anglican kan ik nog geen evaluatie doen op de implementatie criteria.


\vspace{1cm}
{\bf Belangrijkste moeilijkheden:}
Ik ben begonnen aan het implementeren van een model van het spel voordat ik wist hoe inferentie werkt. Dit zorgde er voor dat mijn model slecht was geimplementeerd en weinig functie had. Na het leren van de algoritmes achter inferentie had ik een beter idee hoe mijn model er uit moest zien.

\vspace{1cm}
{\bf Gepland werk:} 
Mijn prioriteiten liggen bij het implementeren van het juiste model in ProbLog en Anglican. Vanaf dat dit gebeurd is zal ik een analyse van het programma maken aan de hand van de implementatie van het systeem ProbLog en Anglican. Aan de hand van deze analyse kan ik dan verschillende criteria bepalen om te evalueren. Tussentijdse teksten worden ook geschreven. In November wil ik de tekst over ProbLog maken en in december de tekst over Anglican. De tekst over de evaluatiecriteria en resulataten zullen volgen na ik het model heb geschreven in de 2 talen.


\vspace{1cm}
{\bf Als ik verder werk zoals ik tot nu toe deed, dan denk ik 12/20 te
    verdienen op het einde.}

{\bf Ik plan mijn eindwerk af te geven in juni.} 


\end{document}
